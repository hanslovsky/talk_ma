    \begin{scope}
        \node (t1) {\huge $t$};
        \node[hypotheses_one_object, below=of t1, circle, draw] (x11) {$X_1^t$};
        \node[hypotheses_one_object, below=of x11, circle, draw] (x12) {$X_2^t$};
    \end{scope}
    \begin{scope}[on background layer]
        \node[hypotheses_division_duplicate, below=of x12, circle, draw, dashed] (x13) {$\bar{X}_2^t$};
    \end{scope}
    
    
    \begin{scope}
        \node[right=of t1, xshift=-6mm] (t2) {\huge $t+1$};
        \node[hypotheses_two_objects, below=of t2, circle, draw] (x21) {$X_3^{t+1}$};
        \node[hypotheses_one_object, below=of x21, circle, draw] (x22) {$X_{4}^{t+1}$};
    \end{scope}
    \begin{scope}[on background layer]
        \node[hypothesesdetection, below=of x22, circle, draw] (x23)  {$X_4^{t+1}$};
    \end{scope}
    
    \begin{scope}
        \node[right=of t2, xshift=-6mm] (t3) {\huge $t+2$};
        \node[hypotheses_one_object, below=of t3, circle, draw] (x31) {$X_5^{t+2}$};
        \node[hypotheses_one_object, below=of x31, circle, draw] (x32) {$X_6^{t+2}$};
        \node[hypotheses_one_object, below=of x32, circle, draw] (x33) {$X_7^{t+2}$};
    \end{scope}
    

    \begin{scope}[on background layer]
        \node[rectangle, draw, color=hypothesesbackground!40, fill=hypothesesbackground!30,
        fit=(x11) (x12) (x13), inner sep=6mm] (b1) {};
        \node[rectangle, draw, color=hypothesesbackground!40, fill=hypothesesbackground!30,
        fit=(x21) (x22) (x23), inner sep=6mm] (b2) {};
        \node[rectangle, draw, color=hypothesesbackground!40, fill=hypothesesbackground!30,
        fit=(x31) (x32) (x33), inner sep=6mm] (b3) {};
    \end{scope}

    \path[hypothesestransition] (x11) edge (x21);
    \path[hypothesestransition] (x12) edge (x21);
    \path[hypothesestransition] (x12) edge (x22);

    \path[hypothesestransition] (x21) edge (x31);
    \path[hypothesestransition] (x21) edge (x32);
    \path[hypothesestransition] (x22) edge (x33);


%%% Local Variables: 
%%% mode: latex
%%% TeX-master: "../../../main"
%%% End: 
