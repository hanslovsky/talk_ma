\section{Cell Tracking}

\begin{frame}{Cell Tracking - Requirements}
    \begin{figure}
        \centering
        \begin{tikzpicture}
            \begin{scope}[baseline=(image1.south)]
                \node[label=above:$t$] (image1) {
                    \includegraphics[width=0.25\textwidth]{images/cell_tracking/example_00.png}
                };
                \uncover<2->{
                    \node[transparent_node, red, xshift=-12pt, yshift=5pt] (d11) at (image1.center) {}; 
                    \node[transparent_node, green, xshift=20pt, yshift=-5pt] (d12) at (image1.center) {}; 
                }
            \end{scope}
            \begin{scope}[baseline=(image2.south)]
                \node[right=of image1, xshift=-20pt, label=above:$t+1$] (image2) {
                    \includegraphics[width=0.25\textwidth]{images/cell_tracking/example_01.png}
                };
                \uncover<3->{
                    \node[transparent_node, red, xshift=-25pt, yshift=15pt] (d21) at (image2.center) {}; 
                    \node[transparent_node, red, xshift=0pt, yshift=-5pt] (d22) at (image2.center) {};
                    \node[transparent_node, green, xshift=18pt, yshift=-10pt] (d23) at (image2.center) {};
                    \path[assignment_arrow, red, bend left=20] (d11) edge (d21);
                    \path[assignment_arrow, red, bend left=10] (d11) edge (d22);
                    \path[assignment_arrow, green, bend right=20] (d12) edge (d23.south west);
                }
            \end{scope}
            \begin{scope}[baseline=(image3.south)]
                \node[right=of image2, xshift=-20pt, label=above:$t+2$] (image3) {
                    \includegraphics[width=0.25\textwidth]{images/cell_tracking/example_02.png}
                };
                \uncover<4->{
                    \node[transparent_node, red, xshift=-26pt, yshift=13pt] (d31) at (image3.center) {}; 
                    \node[transparent_node, red, xshift=-7pt, yshift=-13pt] (d32) at (image3.center) {};
                    \node[transparent_node, green, xshift=18pt, yshift=-8pt] (d33) at (image3.center) {}; 
                    \path[assignment_arrow, red, bend left=20] (d21) edge (d31);
                    \path[assignment_arrow, red, bend left=20] (d22.north east) edge (d32);
                    \path[assignment_arrow, green, bend right=35] (d23.south east) edge (d33.south west);
                }
            \end{scope}
        \end{tikzpicture}
    \end{figure}
            
    \begin{itemize}
          \item Extract cell lineages from embryo data.
          \item Reconstruct fates of all cells in the embryo.
          \item Applicable in 2D and 3D.
    \end{itemize}
\end{frame}


\begin{frame}{Tracking-by-Assignment}
    \begin{figure}
        \centering
        \includegraphics[width=\textwidth]{images/cell_tracking/pipeline_mod.png}
        \label{fig:chaingraph-pipeline}
    \end{figure}
\end{frame}

\begin{frame}{Hypotheses Graph}
    % \begin{figure}
    \begin{center}%
        \only<1-3>{%
            % \begin{subfigure}[b]{0.44\textwidth}
            %     \centering
                \scalebox{0.92}{
                    \begin{tikzpicture}[minimum size=58pt,scale=0.45, every node/.style={scale=0.45, text=black, font=\LARGE}, thick]%
                        \input{images/cell_tracking/hypotheses-graph.tex}%
                        \begin{scope}[overlay]% ignore this scope for bounding box calculation
                        \visible<2-3>{
                            \node[arrowstyle=2, rotate=30, below left=of x13, yshift=60, xshift=10]{Is this a cell?};
                        }
                        \visible<3>{
                            \node[arrowstyle=2, rotate=160, xshift=-100]
                            at ($(x13.south east)!0.5!(x22.south east)$)
                            {\rotatebox{180}{Object $3$ moving to object $5$?}};
                        }
                        \end{scope}%
                    \end{tikzpicture}
                }
            % \end{subfigure}
        }%
        % \hfill
        \only<4->{
            % \begin{subfigure}[b]{0.44\textwidth}
            %     \centering
                \scalebox{0.92}{
                    \begin{tikzpicture}[minimum size=58pt,scale=0.45, every node/.style={scale=0.45, text=black, font=\LARGE}, thick]
                        \input{images/cell_tracking/hypotheses-graph-after-inference.tex}%
                        
                    \end{tikzpicture}
                }
            % \end{subfigure}
            }
        \end{center}
    %     \caption{Hypotheses Graph for potential detections and assignments.}
    %     \label{fig:cell-tracking-hypotheses}
    % \end{figure}
\end{frame}
