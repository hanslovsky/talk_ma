\section{Joint Segmentation and Tracking}

\begin{frame}{Motivation}
    \begin{figure}
    \centering
    \begin{subfigure}[t]{0.48\textwidth}
        \includegraphics[width=\textwidth]{images/joint/mitocheck_255_max.pdf}
        \caption{}
        \label{fig:joint-underseg-no-detection}
    \end{subfigure}
    \hfill
    \begin{subfigure}[t]{0.48\textwidth}
        \includegraphics[width=\textwidth]{images/joint/mitocheck_030_max.pdf}
        \caption{}
        \label{fig:joint-underseg-mergers}
    \end{subfigure}
    \caption{Motivating the joint segmentation and tracking method.}
    \label{fig:joint-motivation-example}
\end{figure}
\end{frame}

\begin{frame}
    \frametitle{Workflow}
    \begin{figure}
        \centering
        \scalebox{0.65}{
            \input{images/joint/pipeline.tex}
        }
        \caption{Joint segmentation and tracking workflow.}
        \label{fig:joint-pipeline}
    \end{figure}
\end{frame}

\begin{frame}
    \frametitle{Definitions}
    \begin{itemize}
          \item \emph{Segment} or \emph{superpixel/-voxel}: Smallest unit in a segmentation
          \item \emph{Region}: Any single segment or union of two neighboring regions
          \item \emph{Conflict}: Two regions that contradict each other
          \item \emph{Connected Component}: A region that has no neighbors (is completely surrounded
        by background)
    \end{itemize}
\end{frame}

\begin{frame}
    \frametitle{Definitions - Example}

\end{frame}

\begin{frame}
    \frametitle{Graphical Model}
    BILD VOM GM
\end{frame}

\begin{frame}
    \frametitle{Graphical Model - Details}
    einige Slides zur Mathematik
\end{frame}

\begin{frame}
    \frametitle{Experiments}
    einige Slides zu den Experimenten
\end{frame}


%%% Local Variables: 
%%% mode: latex
%%% TeX-master: "../main"
%%% End: 
