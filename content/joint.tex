\section{Joint Segmentation and Tracking}

\begin{frame}{Motivation}
    \begin{figure}
        \centering
        \begin{subfigure}[t]{0.48\textwidth}
            \includegraphics[width=\textwidth]{images/joint/mitocheck_255_max.pdf}
            \caption{Missed cells}
            \label{fig:joint-underseg-no-detection}
        \end{subfigure}
        \hfill
        \begin{subfigure}[t]{0.48\textwidth}
            \includegraphics[width=\textwidth]{images/joint/mitocheck_030_max.pdf}
            \caption{Merged objects}
            \label{fig:joint-underseg-mergers}
        \end{subfigure}
    \end{figure}

\end{frame}

\begin{frame}
    \frametitle{Workflow}
    \begin{figure}
        \centering
        \scalebox{0.65}{
            \input{images/joint/pipeline.tex}
        }
        \caption{Joint segmentation and tracking workflow.}
        \label{fig:joint-pipeline}
    \end{figure}
\end{frame}

% \begin{frame}
%     \frametitle{Definitions}
%     \begin{itemize}
%           \item \emph{Segment} or \emph{superpixel/-voxel}: Smallest unit in a segmentation
%           \item \emph{Region}: Any single segment or union of two neighboring regions
%           \item \emph{Conflict}: Set of overlapping and thus contradicting regions
%           \item \emph{Connected Component}: A region that has no neighbors (is completely surrounded
%         by background)
%     \end{itemize}
% \end{frame}

\begin{frame}
    \frametitle{Definitions - Example}
    \begin{minipage}[t][0.4\textheight][t]{\textwidth}%
        \begin{figure}%
            \centering%
            \begin{subfigure}[t]{0.23\textwidth}%
                \centering%
                \includegraphics[width=\textwidth]{images/joint/overseg/75/02/raw_contrast.png}
            \end{subfigure}%
            \hfill
            \begin{subfigure}[t]{0.23\textwidth}
                \centering
                \begin{tikzpicture}[baseline=(image1.south)]
                    {\uncover<2->{\node[anchor=south west,inner sep=0] (image1) {
                                \includegraphics[width=\textwidth]{images/joint/overseg/75/02/colored00.png}};
                            \begin{scope}[x={(image1.south east)},y={(image1.north west)}]
                                \node[region_id] at (0.4, 0.75) {\huge{$3$}};
                                \node[region_id] at (0.65, 0.7) {\huge{$2$}};
                                \node[region_id] at (0.5, 0.32) {\huge{$1$}};
                            \end{scope}}}%
                \end{tikzpicture}%
            \end{subfigure}%
            \hfill%
            \begin{subfigure}[t]{0.23\textwidth}
                \centering
                \begin{tikzpicture}[baseline=(image2.south)]
                    {\uncover<3->{\node[anchor=south west,inner sep=0] (image2) {
                                \includegraphics[width=\textwidth]{images/joint/overseg/75/02/colored01_all.png}};
                            \begin{scope}[x={(image1.south east)},y={(image2.north west)}]
                                \node[region_id] at (0.53, 0.73) {\huge{$4$}};
                                \node[region_id] at (0.5, 0.32) {\huge{$1$}};
                            \end{scope}}}
                \end{tikzpicture}%
            \end{subfigure}%
            \hfill%
            \begin{subfigure}[t]{0.23\textwidth}
                \centering
                \begin{tikzpicture}[baseline=(image3.south)]
                    {\uncover<4->{\node[anchor=south west,inner sep=0] (image3) {
                                \includegraphics[width=\textwidth]{images/joint/overseg/75/02/colored02.png}};
                            \begin{scope}[x={(image1.south east)},y={(image3.north west)}]
                                \node[region_id] at (0.5, 0.32) {\huge{$5$}};
                            \end{scope}}}
                \end{tikzpicture}%
            \end{subfigure}%
            \label{fig:joint-segments-example}
        \end{figure}%
    \end{minipage}%
    \begin{overprint}%
        %\only<1-4>{
        \begin{itemize}%
              \item<2-4|only@2-4> Segments: $\{1,2,3\}$%
              \item<2|only@2> Regions: $\{1,2,3\}$%
              \item<3|only@3> Regions: $\{1,2,3,4\}$%
              \item<4|only@4> Regions: $\{1,2,3,4,5\}$%
              \item<4> Connected Component: $\{5\}$%
            %   \item<3|only@3> Conflicts: $\{1,4\},\{2,4\}$%
              \item<4> Conflicts: $\{2,4,5\},\{3,4,5\},\{1,5\}$%
        \end{itemize}%}%
        % \vspace{-50pt}
        \visible<5-|only@5->{
            \begin{figure}
                \centering
                \begin{subfigure}[b]{0.40\textwidth}
                    \begin{flushleft}
                    \scalebox{0.950}{
                    \begin{tikzpicture}[baseline=(bl)]
                        \node[conflict_graph] (r5) {$5$};
                        \node[conflict_graph, right=of r5] (r1) {$3$};
                        \node[conflict_graph, below=of r5] (r2) {$2$};
                        \node[conflict_graph, right=of r2] (r4) {$4$};
                        \node[conflict_graph, left=of r2] (r3) {$1$};
                        \path[conflict_edge] (r5) edge (r1);
                        \path[conflict_edge] (r5) edge (r2);
                        \path[conflict_edge] (r5) edge (r3);
                        \path[conflict_edge] (r5) edge (r4);
                        \path[conflict_edge] (r4) edge (r1);
                        \path[conflict_edge] (r4) edge (r2);
                        \coordinate[below=of r1.south] (bl);
                    \end{tikzpicture}}
                    \end{flushleft}
                \end{subfigure}%
                %\hfill%
                \visible<6-|only@6->{
                    \scalebox{0.7}{
                    \tikzfancyarrow{Factor Graph}}%
                %\hfill%
                \begin{subfigure}[b]{0.40\textwidth}
                    \centering
                    \begin{flushright}
                    \scalebox{0.950}{
                    \begin{tikzpicture}[baseline=(bl)]
                        \node[conflict_graph] (r5) {$5$};
                        \node[conflict_graph, right=of r5] (r1) {$3$};
                        \node[conflict_graph, below=of r5] (r2) {$2$};
                        \node[conflict_graph, right=of r2] (r4) {$4$};
                        \node[conflict_graph, left=of r2] (r3) {$1$};
                        \node[conflict] (c15) at ($(r3)!0.5!(r5)$) {};
                        \node[conflict, xshift=9, yshift=9] (c245) at (r2.north east) {};
                        \node[conflict, xshift=-9, yshift=-9] (c345) at (r1.south west) {};
                        \path[conflict_edge] (r5) edge (c15);
                        \path[conflict_edge] (r3) edge (c15);
                        \path[conflict_edge] (r5) edge (c245);
                        \path[conflict_edge] (r4) edge (c245);
                        \path[conflict_edge] (r2) edge (c245);
                        \path[conflict_edge] (r5) edge (c345);
                        \path[conflict_edge] (r4) edge (c345);
                        \path[conflict_edge] (r1) edge (c345);
                        \coordinate[below=of r1.south] (bl);
                    \end{tikzpicture}}
                \end{flushright}%
                \end{subfigure}}%
            \end{figure}
        }%
    \end{overprint}%
\end{frame}


\begin{frame}
    \frametitle{From Regions to Factor Graph}
    \begin{minipage}[t][0.4\textheight][t]{\textwidth}%
        \begin{figure}
            \centering
            \begin{subfigure}[t]{0.23\textwidth}
                \centering
                \includegraphics[width=\textwidth]{images/joint/overseg/75/02/raw_contrast.png}
                % \caption{Raw data}
            \end{subfigure}%
            \hfill%
            \begin{subfigure}[t]{0.23\textwidth}
                \centering
                \begin{tikzpicture}[baseline=(image1.south)]
                    \node[anchor=south west,inner sep=0] (image1) {
                        \includegraphics[width=\textwidth]{images/joint/overseg/75/02/colored00.png}};
                    \begin{scope}[x={(image1.south east)},y={(image1.north west)}]
                        \node[region_id] at (0.4, 0.75) {\huge{$3$}};
                        \node[region_id] at (0.65, 0.7) {\huge{$2$}};
                        \node[region_id] at (0.5, 0.32) {\huge{$1$}};
                    \end{scope}
                \end{tikzpicture}%
            \end{subfigure}%
            \hfill%
            \begin{subfigure}[t]{0.23\textwidth}
                \centering
                \begin{tikzpicture}[baseline=(image2.south)]
                    \node[anchor=south west,inner sep=0] (image2) {
                        \includegraphics[width=\textwidth]{images/joint/overseg/75/02/colored01_all.png}};
                    \begin{scope}[x={(image1.south east)},y={(image2.north west)}]
                        \node[region_id] at (0.53, 0.73) {\huge{$4$}};
                        \node[region_id] at (0.5, 0.32) {\huge{$1$}};
                    \end{scope}
                \end{tikzpicture}%
            \end{subfigure}%
            \hfill%
            \begin{subfigure}[t]{0.23\textwidth}
                \centering
                \begin{tikzpicture}[baseline=(image3.south)]
                    \node[anchor=south west,inner sep=0] (image3) {
                        \includegraphics[width=\textwidth]{images/joint/overseg/75/02/colored02.png}};
                    \begin{scope}[x={(image1.south east)},y={(image3.north west)}]
                        \node[region_id] at (0.5, 0.32) {\huge{$5$}};
                    \end{scope}
                \end{tikzpicture}%
            \end{subfigure}%
        \end{figure}
    \end{minipage}%
    % 
    \begin{figure}
        \centering
        \newcommand{\scalingfactor}{0.3}
        \scalebox{0.9}{
            \begin{tikzpicture}
                \begin{scope}
    \begin{scope}[
        every node/.append style={yslant=-0.5,xslant=1},
        yslant=-0.5,xslant=1]
        \node[inner sep=0] (image1) {
            \phantom{\includegraphics[width=\scalingfactor\textwidth]{images/joint/78_seg_crop.png}}
        };
        \begin{pgfonlayer}{upper}
            \begin{scope}[every node/.append style={scale=0.7}]
                \node[fg_det, label={[font=\tiny]center:$X_1^t$}] (a1) at (image1.south east) {};
                \node[fg_det, label={[font=\tiny]center:$X_2^t$}] (a2) at ($(image1.south east)!0.5!(image1.north east)$) {};
                \node[fg_det, label={[font=\tiny]center:$X_3^t$}] (a3) at (image1.north east) {};
                \node[fg_det, label={[font=\tiny]center:$X_5^t$}] (a5) at ($(image1.south west)!0.5!(image1.north west)$) {};
                \node[fg_det, label={[font=\tiny]center:$X_4^t$}] (a4) at ($(a3)!0.5!(a5)$) {};
            \end{scope}
            \begin{scope}[every node/.append style={scale=0.65}]
                \node[conflict,yshift=-5] (c1) at ($(a1)!0.5!(a5)$) {};
                \node[conflict, right=of a3, xshift=-10, yshift=-10] (c2) {};
                \node[conflict, yshift=-20] (c3) at (a4) {};
                \node[count, yshift=-20] (c4)  at ($(a2)!0.5!(a4)$) {};
                
                \path[count] (a1) edge (c4);
                \path[count] (a2) edge (c4);
                \path[count] (a3.south) edge (c4);
                \path[count] (a4) edge (c4);
                \path[count] (a5) edge[bend right=20] (c4);
                
                \path[conflict] (a5) edge[bend right=10] (c1);
                \path[conflict] (a1) edge (c1);

                \path[conflict] (a2) edge (c2);
                \path[conflict] (a4) edge[bend left=50] (c2);
                \path[conflict] (a5) edge[bend left=65] coordinate[pos=0.65](helpercoord1) (c2);

                \path[conflict] (a3) edge[bend left=10] (c3);
                \path[conflict] (a4) edge (c3);
                \path[conflict] (a5) edge (c3);
            \end{scope}
        \end{pgfonlayer}
        \begin{pgfonlayer}{bgupper}
            \node[rectangle, color=black,thick, fill=hypothesesbackground!30, opacity=0.8, draw=black,
            fit=(a1) (c2) (a5) (helpercoord1), inner sep=2, opacity=0.8, label={[xshift=5]above:{}}]
            (bgup) {};
        \end{pgfonlayer}
    \end{scope}
\end{scope}


%%% Local Variables: 
%%% mode: latex
%%% TeX-master: "../../main"
%%% End: 

            \end{tikzpicture}
        }
    \end{figure}
\end{frame}


\begin{frame}
    \frametitle{Graphical Model}
    \centering
    \newcommand{\distancebetweenlayers}{90}
    \newcommand{\scalingfactor}{0.3}
    \scalebox{0.8}{
        \begin{tikzpicture}
            \input{images/joint/graph_detailed.tex}
        \end{tikzpicture}
    }\hfill%
    \scalebox{0.8}{
        \begin{tikzpicture}%
            \visible<1->{
                \begin{scope}
                    \node[circle, draw, thick] (a11) {$1$};
                    \node[circle, draw, thick, right=of a11, xshift=-15] (a12) {$2$};
                    \node[circle, draw, thick, right=of a12, xshift=-15] (a13) {$3$};
                    \node[circle, draw, thick, above=of a12, yshift=10] (a15) {$5$};
                    \node[circle, draw, thick] (a14) at ($(a15)!0.5!(a13)$) {$4$};
                    \path (a11) edge (a15);
                    \path (a14) edge (a15);
                    \path (a12) edge (a14);
                    \path (a13) edge (a14);
                \end{scope}
            }%
            \visible<2->{
                \begin{scope}[yshift=-120]
                    \node[circle, draw, thick] (a21) {$6$};
                    \node[circle, draw, thick, opacity=0.0, right=of a21, xshift=-15] (fake1) {$6$};
                    \node[circle, draw, thick, right=of fake1, xshift=-15] (a22) {$7$};
                    \node[circle, draw, thick, above=of fake1, yshift=10] (a23) {$8$};
                    \path (a21) edge (a23);
                    \path (a22) edge (a23);
                \end{scope}
            }%
        \end{tikzpicture}}%
    \vfill
    % \caption{Factor Graph}
    \begin{minipage}{0.33\textwidth}
        \begin{tabular}{ll}
            \tikz[baseline, inner sep=1pt]\node[scale=0.4, thick, draw, circle, fg_det, text=black] at
            (0,2.5pt){$X$}; & Detection \\
            \uncover<3->{\tikz[baseline, inner sep=1pt]\node[scale=0.4, thick, draw, circle, fg_tra, text=black] at
                (0,2.5pt){$Y$};} & \uncover<3->{Assignment} \\
        \end{tabular}
    \end{minipage}%
    \begin{minipage}{0.33\textwidth}
        \begin{tabular}{ll}
            \tikz[baseline, inner sep=1pt]{\node[scale=2.5, thick, draw, circle, conflict] (c1) at
                (0,2.8pt){};
                \node[opacity=0.0, fg_det, scale=0.4] (fake) at (c1.center) {\phantom{$X$}};
                \path[conflict] (fake.west) edge (fake.east);
            }
            & Conflict \\
            \uncover<3->{\tikz[baseline, inner sep=1pt]{\node[scale=2.5, thick, draw, rectangle, transfac] (c1) at
                    (0,2.8pt){};
                    \node[opacity=0.0, fg_det, scale=0.4] (fake) at (c1.center) {\phantom{$X$}};
                    \path[transfac] (fake.west) edge (fake.east);
                }
            }
            & \uncover<3->{Outgoing} \\
        \end{tabular}
    \end{minipage}%
    \begin{minipage}{0.33\textwidth}
        \begin{tabular}{ll}
            \tikz[baseline, inner sep=1pt]{\node[scale=2.5, thick, draw, count] (c1) at
                (0,2.8pt){};
                \node[opacity=0.0, fg_det, scale=0.4] (fake) at (c1.center) {\phantom{$X$}};
                \path[count] (fake.west) edge (fake.east);
            }
            & Count \\
            \uncover<3->{\tikz[baseline, inner sep=1pt]{\node[scale=2.5, thick, draw, rectangle, transfac] (c1) at
                    (0,2.8pt){};
                    \node[opacity=0.0, fg_det, scale=0.4] (fake) at (c1.center) {\phantom{$X$}};
                    \path[transfac] (fake.west) edge (fake.east);
                }
            }
            & \uncover<3->{Incoming} \\
        \end{tabular}
    \end{minipage}
\end{frame}

\begin{frame}{Graphical Model - Wrap-Up}
    \newsavebox{\detVar}
    \savebox{\detVar}{\tikz[baseline, inner sep=1pt]\node[scale=0.4, thick, draw, circle, fg_det, text=black] at
        (0,2.5pt){$X$};}
    \newsavebox{\transVar}
    \savebox{\transVar}{\tikz[baseline, inner sep=1pt]\node[scale=0.4, thick, draw, circle, fg_tra, text=black] at
        (0,2.5pt){$Y$};}
    \newsavebox{\detFac}
    \savebox{\detFac}{\tikz[baseline, inner sep=1pt]{\node[scale=2.5, thick, draw, circle, conflict] (c1) at
            (0,2.8pt){};
            \node[opacity=0.0, fg_det, scale=0.4] (fake) at (c1.center) {\phantom{$X$}};
            \path[conflict] (fake.west) edge (fake.east);
        }}
    \newsavebox{\countFac}
    \savebox{\countFac}{\tikz[baseline, inner sep=1pt]{\node[scale=2.5, thick, draw, circle, count] (c1) at
            (0,2.8pt){};
            \node[opacity=0.0, fg_det, scale=0.4] (fake) at (c1.center) {\phantom{$X$}};
            \path[count] (fake.west) edge (fake.east);
        }}
    \newsavebox{\outFac}
    \savebox{\outFac}{\tikz[baseline, inner sep=1pt]{\node[scale=2.5, thick, draw, circle, transfac] (c1) at
            (0,2.8pt){};
            \node[opacity=0.0, fg_det, scale=0.4] (fake) at (c1.center) {\phantom{$X$}};
            \path[transfac] (fake.west) edge (fake.east);
        }}
    \newsavebox{\inFac}
    \savebox{\inFac}{\tikz[baseline, inner sep=1pt]{\node[scale=2.5, thick, draw, circle, transfac] (c1) at
            (0,2.8pt){};
            \node[opacity=0.0, fg_det, scale=0.4] (fake) at (c1.center) {\phantom{$X$}};
            \path[transfac] (fake.west) edge (fake.east);
        }}
    \begin{minipage}{0.71\textwidth}
        \begin{itemize}
              \item Random Variables
            \begin{itemize}
                  \item[\usebox{\detVar}] Detection
                  \item[\usebox{\transVar}] Assignment
            \end{itemize}
              \item Factors
            \begin{itemize}
                  \item[\usebox{\detFac}] Conflict or Detection factor $\psi_{\text{det}}$
                \begin{itemize}
                      \item[] Detection priors
                      \item[] Conflict constraints
                \end{itemize}
                  \item[\usebox{\countFac}] Count Factor $\psi_{\text{count}}$
                \begin{itemize}
                      \item[] Count prior
                \end{itemize}
                  \item[\usebox{\outFac}] Outgoing Factor $\psi_{\text{out}}$
                \begin{itemize}
                      \item[] Move and Division priors
                      \item[] Assignment Consistency Constraints
                \end{itemize}
                  \item[\usebox{\inFac}] Incoming Factor $\psi_{\text{in}}$
                \begin{itemize}
                      \item[] Assignment Consistency Constraints
                \end{itemize}
            \end{itemize}
        \end{itemize}
    \end{minipage}
    \begin{minipage}{0.25\textwidth}
        \newcommand{\distancebetweenlayers}{60}
        \newcommand{\scalingfactor}{0.8}
        \scalebox{0.55}{
            \begin{tikzpicture}[every node/.append style={scale=0.98}]
                \input{images/joint/graph_detailed_at_once.tex}
            \end{tikzpicture}
        }\hfill
        \vfill
    \end{minipage}
\end{frame}

\begin{frame}[shrink=25]
    \frametitle{Graphical Model - Joint Probability}
    \begin{align*}
        P(\mathcal{X},\mathcal{Y}) = \frac{1}{Z}\prod_t\prod_i
        \Bigg(&\psi_{\text{count}}(\{X_{i\alpha}^t\}_{\alpha \in I_i^t},f_{i}^t) \\ \nonumber
        &\times\prod_{\alpha}\Big(\psi_{\mathrm{out}}(X_{i\alpha}^t, \mathcal{Y}_{i\alpha\rightarrow}^{t},
        f_{i\alpha}^t, \Xi_{i\alpha\rightarrow}^{t}) \cdot \psi_{\mathrm{in}}(X_{i\alpha}^{t},
        \mathcal{Y}_{\rightarrow i\alpha}^{t-1})\Big) \\ \nonumber
        &\times\prod_{\nu}\Big(\psi_{\text{det}}(\mathcal{K}_{i\nu}^t, \xi_{i\nu}^t)\Big)\Bigg)
    \end{align*}
    \vspace{10pt}
    \begin{center}
        \Large{$\Rightarrow$ Find MAP solution with integer linear programming.}
    \end{center}
\end{frame}

% \begin{frame}
%     \frametitle{Classifiers}
%     Details zu den Classifiern
% \end{frame}

\begin{frame}
    \frametitle{Experiments}
    \begin{figure}
        \centering
        \begin{tikzpicture}
            \node[label=above:{$t=74$}] (seg1) {\includegraphics[width=0.2\textwidth]{images/joint/res/sequence_01/overseg_raw74.png}};
            \node[label=above:{$t=75$}, right=of seg1.center] (seg2) {\includegraphics[width=0.2\textwidth]{images/joint/res/sequence_01/overseg_raw75.png}};
            \node[label=above:{$t=76$}, right=of seg2.center] (seg3) {\includegraphics[width=0.2\textwidth]{images/joint/res/sequence_01/overseg_raw76.png}};
            \node[align=center, left=of seg1.center] {Over- \\ segmentation};
            \uncover<2->{
                \node[below=of seg1.center] (tra1) {\includegraphics[width=0.2\textwidth]{images/joint/res/sequence_01/ov_raw74.png}};
                \node[below=of seg2.center] (tra2) {\includegraphics[width=0.2\textwidth]{images/joint/res/sequence_01/ov_raw75.png}};
                \node[below=of seg3.center] (tra3) {\includegraphics[width=0.2\textwidth]{images/joint/res/sequence_01/ov_raw76.png}};
                \node[align=center, left=of tra1.center] {Tracking};
            }
        \end{tikzpicture}
    \end{figure}
\end{frame}

\begin{frame}
    \frametitle{Experiments}
    \begin{figure}
        \centering
        \begin{tikzpicture}
            \node[label=above:{$t=82$}] (seg1) {\includegraphics[width=0.2\textwidth]{images/joint/res/sequence_02/overseg_raw82.png}};
            \node[label=above:{$t=83$}, right=of seg1.center] (seg2) {\includegraphics[width=0.2\textwidth]{images/joint/res/sequence_02/overseg_raw83.png}};
            \node[label=above:{$t=84$}, right=of seg2.center] (seg3) {\includegraphics[width=0.2\textwidth]{images/joint/res/sequence_02/overseg_raw84.png}};
            \node[align=center, left=of seg1.center] {Over- \\ segmentation};
            \uncover<2->{
                \node[below=of seg1.center] (tra1) {\includegraphics[width=0.2\textwidth]{images/joint/res/sequence_02/ov_raw82.png}};
                \node[below=of seg2.center] (tra2) {\includegraphics[width=0.2\textwidth]{images/joint/res/sequence_02/ov_raw83.png}};
                \node[below=of seg3.center] (tra3) {\includegraphics[width=0.2\textwidth]{images/joint/res/sequence_02/ov_raw84.png}};
                \node[align=center, left=of tra1.center] {Tracking};
            }
        \end{tikzpicture}
    \end{figure}
\end{frame}

% \begin{frame}[shrink=25]
%     \frametitle{Experiments}
%     \begin{figure}
%         \centering
%         \begin{tikzpicture}
%             \node[label={[yshift=-2.5mm]above:\small{Segmentation}}] (seg1) {\includegraphics[width=0.32\textwidth]{images/joint/res/sequence_03/00_colored00_raw.png}};
%             \node[label={[yshift=-2.5mm]above:\small{Region Merging}}, right=of seg1] (seg2) {\includegraphics[width=0.32\textwidth]{images/joint/res/sequence_03/00_colored01_raw.png}};
%             \node[label={[yshift=-2.5mm]above:\small{Region Merging}}, below=of seg1, yshift=7.5mm] (seg3) {\includegraphics[width=0.32\textwidth]{images/joint/res/sequence_03/00_colored02_raw.png}};
%             \node[label={[yshift=-2.5mm]above:\small{Region Merging}}, right=of seg3] (seg4) {\includegraphics[width=0.32\textwidth]{images/joint/res/sequence_03/00_colored03_raw.png}};
%             \node[label={[yshift=-2.5mm]above:\small{Region Merging}}, below=of seg3, yshift=7.5mm] (seg5)  {\includegraphics[width=0.32\textwidth]{images/joint/res/sequence_03/00_colored04_raw.png}};
%             \node[label={[yshift=-2.5mm]above:\small{Tracking}}, right=of seg5] (tra1) {\includegraphics[width=0.32\textwidth]{images/joint/res/sequence_03/00_track_raw.png}};
%         \end{tikzpicture}
%     \end{figure}
% \end{frame}

% \begin{frame}
%     \frametitle{Open Challenges}
%     \renewcommand{\arraystretch}{1.5}
%     \begin{tabularx}{\textwidth}{X|X}
%         Challenge & Solution \\ \hline
%         Region merging errors propagated into tracking result  & Improve region merging (vigra) \\
%         Tractability & Change hypotheses generation \\
%         Parameters not optimal & Do exhaustive grid search \\
%     \end{tabularx}
% \end{frame}


\begin{frame}
    \frametitle{Limitations of Joint Segmentation and Tracking}
    \begin{itemize}
          \item<1-> Errors in the initial oversegmentation cannot be repaired by the tracking.
          \item<2-> Tractability: A finer oversegmentation results in a larger number of random variables
        and higher order factors.
    \end{itemize}
\end{frame}


\begin{frame}
    \frametitle{Summary}
    \begin{itemize}
          \item<1-> Merger resolution for conservation tracking
        \begin{itemize}
              \item Post-processing step to prevent loss of tracks of merged objects
              \item Reconstruct cell identities according to conservation tracking result
        \end{itemize}
          \item<2-> Joint segmentation and tracking
        \begin{itemize}
              \item Jointly optimize tracking and choose the best segmentation from multiple
            competing hypotheses
              \item Tracking in challenging data possible
        \end{itemize}
    \end{itemize}
\end{frame}


\begin{frame}
    \begin{center}
        Thank you!
    \end{center}
\end{frame}


%%% Local Variables: 
%%% mode: latex
%%% TeX-master: "../main"
%%% End: 
