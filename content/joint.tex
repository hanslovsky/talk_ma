\section{Joint Segmentation and Tracking}

\begin{frame}{Motivation}
    \begin{figure}
    \centering
    \begin{subfigure}[t]{0.48\textwidth}
        \includegraphics[width=\textwidth]{images/joint/mitocheck_255_max.pdf}
        \caption{}
        \label{fig:joint-underseg-no-detection}
    \end{subfigure}
    \hfill
    \begin{subfigure}[t]{0.48\textwidth}
        \includegraphics[width=\textwidth]{images/joint/mitocheck_030_max.pdf}
        \caption{}
        \label{fig:joint-underseg-mergers}
    \end{subfigure}
    \caption{Motivating the joint segmentation and tracking method.}
    \label{fig:joint-motivation-example}
\end{figure}
\end{frame}

\begin{frame}
    \frametitle{Workflow}
    \begin{figure}
        \centering
        \scalebox{0.65}{
            \input{images/joint/pipeline.tex}
        }
        \caption{Joint segmentation and tracking workflow.}
        \label{fig:joint-pipeline}
    \end{figure}
\end{frame}

\begin{frame}
    \frametitle{Definitions}
    \begin{itemize}
          \item \emph{Segment} or \emph{superpixel/-voxel}: Smallest unit in a segmentation
          \item \emph{Region}: Any single segment or union of two neighboring regions
          \item \emph{Conflict}: Set of overlapping and thus contradicting regions
          \item \emph{Connected Component}: A region that has no neighbors (is completely surrounded
        by background)
    \end{itemize}
\end{frame}

\begin{frame}
    \frametitle{Definitions - Example}
    \begin{figure}
        \centering
        \begin{subfigure}[t]{0.23\textwidth}
            \centering
            \includegraphics[width=\textwidth]{images/joint/overseg/75/02/raw_contrast.png}
            % \caption{Raw data}
        \end{subfigure}
        ~
        \begin{subfigure}[t]{0.23\textwidth}
            \centering
            
                \begin{tikzpicture}[baseline=(image1.south)]
                    {\uncover<2->{\node[anchor=south west,inner sep=0] (image1) {
                        \includegraphics[width=\textwidth]{images/joint/overseg/75/02/colored00.png}};
                    \begin{scope}[x={(image1.south east)},y={(image1.north west)}]
                        \node[region_id] at (0.4, 0.75) {\huge{$1$}};
                        \node[region_id] at (0.65, 0.7) {\huge{$2$}};
                        \node[region_id] at (0.5, 0.32) {\huge{$3$}};
                    \end{scope}}}
                \end{tikzpicture}
            % \caption{Segments}
        \end{subfigure}
        \hfill
        \begin{subfigure}[t]{0.23\textwidth}
            \centering
                \begin{tikzpicture}[baseline=(image2.south)]
                    {\uncover<3->{\node[anchor=south west,inner sep=0] (image2) {
                        \includegraphics[width=\textwidth]{images/joint/overseg/75/02/colored01_all.png}};
                    \begin{scope}[x={(image1.south east)},y={(image2.north west)}]
                        \node[region_id] at (0.53, 0.73) {\huge{$4$}};
                        \node[region_id] at (0.5, 0.32) {\huge{$3$}};
                    \end{scope}}}
                \end{tikzpicture}
            % \caption{Segments}
        \end{subfigure}
        \hfill
        \begin{subfigure}[t]{0.23\textwidth}
            \centering
            \begin{tikzpicture}[baseline=(image3.south)]
                {\uncover<4->{\node[anchor=south west,inner sep=0] (image3) {
                            \includegraphics[width=\textwidth]{images/joint/overseg/75/02/colored02.png}};
                        \begin{scope}[x={(image1.south east)},y={(image3.north west)}]
                            \node[region_id] at (0.5, 0.32) {\huge{$5$}};
                        \end{scope}}}
            \end{tikzpicture}
            % \caption{Segments}
        \end{subfigure}
        % \caption{Definitions.}
        \label{fig:joint-segments-example}
    \end{figure}
    \visible<1-4|only@1-4>{
    \begin{itemize}
          \item<2-4> Segments: $\{1,2,3\}$
          \item<2|only@2> Regions: $\{1,2,3\}$
          \item<3|only@3> Regions: $\{1,2,3,4\}$
          \item<4|only@4> Regions: $\{1,2,3,4,5\}$
          \item<4> Connected Component: $\{5\}$
        % \item<3|only@3> Conflicts: $\{1,4\},\{2,4\}$
          \item<4> Conflicts: $\{1,4,5\},\{2,4,5\},\{3,5\}$
    \end{itemize}}
\vspace{-50pt}
    \visible<5|only@5>{
        \begin{figure}
            \centering
            \begin{subfigure}[b]{0.48\textwidth}
                \centering
                \begin{tikzpicture}[baseline=(r1.south)]
                    \node[region_graph] (r1) {$1$};
                    \node[region_graph, right=of r1.west] (r2) {$2$};
                    \node[region_graph, below=of r1.north] (r3) {$3$};
                    \node[region_graph, right=of r3.west] (r4) {$4$};
                    \node[region_graph, right=of r2.west] (r5) {$5$};
                    \path[region_edge] (r1) edge (r2);
                    \path[region_edge] (r2) edge (r3);
                    \path[region_edge] (r3) edge (r4);
                \end{tikzpicture}
                \caption{Adjacency}
            \end{subfigure}
            ~
            \begin{subfigure}[b]{0.48\textwidth}
                \centering
                \begin{tikzpicture}[baseline=(r1.south)]
                    \node[conflict_graph] (r5) {$5$};
                    \node[conflict_graph, right=of r5.west] (r1) {$1$};
                    \node[conflict_graph, below=of r5.north] (r2) {$2$};
                    \node[conflict_graph, right=of r2.west] (r4) {$4$};
                    \node[conflict_graph, left=of r2.east] (r3) {$3$};
                    \path[conflict_edge] (r5) edge (r1);
                    \path[conflict_edge] (r5) edge (r2);
                    \path[conflict_edge] (r5) edge (r3);
                    \path[conflict_edge] (r5) edge (r4);
                    \path[conflict_edge] (r4) edge (r1);
                    \path[conflict_edge] (r4) edge (r2);
                \end{tikzpicture}
                \caption{Conflicts}
            \end{subfigure}
            
            \caption{Graph Representation}
        \end{figure}
    }
\end{frame}



\begin{frame}
    \frametitle{Graphical Model}
    \begin{figure}
        \centering
        \newcommand{\distancebetweenlayers}{90}
        \newcommand{\scalingfactor}{0.3}
        \scalebox{0.9}{
            \begin{tikzpicture}
                \input{images/joint/graph_detailed.tex}
            \end{tikzpicture}
        }
        \caption{Factor Graph}
        \label{fig:joint-fg}
    \end{figure}
\end{frame}

\begin{frame}
    \frametitle{Graphical Model - Details}
    einige Slides zur Mathematik
\end{frame}

\begin{frame}
    \frametitle{Classifiers}
    Details zu den Classifiern
\end{frame}

\begin{frame}
    \frametitle{Experiments}
    einige Slides zu den Experimenten
\end{frame}


%%% Local Variables: 
%%% mode: latex
%%% TeX-master: "../main"
%%% End: 
