\section{Conservation Tracking}

\begin{frame}{Concept}
    \begin{itemize}
          \item Use multi-state variables instead of binary variables to model multiple objects per
        detection.
          \item Implement mass conservation laws to ensure consistency.
          \item Resolve tracks that contain merge objects.
    \end{itemize}
\end{frame}

\begin{frame}{Workflow}
    \begin{figure}
        \centering
        \begin{tikzpicture}
            \node[anchor=south west, inner sep=0] (image) at (0,0)
            {\includegraphics[width=\textwidth]{images/conservation/pipeline.png}};
            \begin{scope}[x={(image.south east)},y={(image.north west)}]
                \draw[red,line width=2pt,rounded corners, fill=red!20, fill opacity=0.3]
                (0.81,0.20) rectangle (1.003,0.83);
            \end{scope}
        \end{tikzpicture}
        \caption{Conservation Tracking Workflow [SOURCE]}
        \label{fig:conservation-pipeline}
    \end{figure}
\end{frame}

\begin{frame}{Factor Graph}
    \begin{figure}
        \centering
        \includegraphics[width=\textwidth]{images/conservation/factor_graph.png}
        \caption{Graphical model of the conservation tracking. [SOURCE]}
        \label{fig:conservation-fg}
    \end{figure}
\end{frame}

\begin{frame}{Merger Resolution}
    \begin{figure}
        \centering
        \begin{subfigure}[b]{0.44\textwidth}
            \centering
            \scalebox{0.65}{
                \input{images/conservation/convex_combination}
            }
            \caption{Convex Combination.}
        \end{subfigure}
        \hfill
        \begin{subfigure}[b]{0.44\textwidth}
            \centering
            \scalebox{0.74}{
                \input{images/conservation/gmm-gm-plate}
            }
            \caption{GMM - plate notation.}
        \end{subfigure}
        \caption{Gaussian Mixture Model.}
        \label{fig:conservation-gmm}
    \end{figure}
    \begin{align*}
        \label{eq:gmm-normal}
        \mathsmaller{p(v|h)} &= \mathsmaller{\mathcal{N}({v}|\MUH , \SIGMAH )} \\ 
        &\mathsmaller{= \frac{1}{\sqrt{\det(2\pi\SIGMAH )} }\exp\left(-\frac{1}{2}({v}-\MUH )^\intercal
            \left(\SIGMAH\right) ^{-1} ({v}-\MUH )\right)}
    \end{align*}
\end{frame}

\begin{frame}{Merger Resolution}
    \begin{figure}
        \centering
        \scalebox{0.7}{
            \begin{tikzpicture}[minimum size=58pt,scale=0.45, every node/.style={scale=0.45, font=\LARGE}, thick]
                \input{images/conservation/hypotheses_original.tex}
            \end{tikzpicture}
        }
        \caption{Original hypotheses graph.}
        \label{fig:conservation-hyp-inferred}
    \end{figure}
\end{frame}

\begin{frame}{Merger Resolution}
    \begin{figure}
        \centering
        \scalebox{0.7}{
            \begin{tikzpicture}[minimum size=58pt,scale=0.45, every node/.style={scale=0.45, font=\LARGE}, thick]
                \input{images/conservation/hypotheses_after_inference.tex}
            \end{tikzpicture}
        }
        \caption{Inferred configuration of the hypotheses graph.}
        \label{fig:conservation-hyp-inferred}
    \end{figure}
\end{frame}

\begin{frame}{Merger Resolution}
    \begin{figure}
        \centering
        \scalebox{0.7}{
            \begin{tikzpicture}[minimum size=58pt,scale=0.45, every node/.style={scale=0.45, font=\LARGE}, thick]
                \input{images/conservation/hypotheses_subset.tex}
            \end{tikzpicture}
        }
        \caption{Subset of the hypotheses graph that contains all merged objects.}
        \label{fig:conservation-hyp-inferred}
    \end{figure}
\end{frame}

\begin{frame}{Merger Resolution}
    \begin{figure}
        \centering
        \scalebox{0.7}{
            \begin{tikzpicture}[minimum size=58pt,scale=0.45, every node/.style={scale=0.45, font=\LARGE}, thick]
                \input{images/conservation/hypotheses_subset_inferred.tex}
            \end{tikzpicture}
        }
        \caption{Inferred configuration of the subset.}
        \label{fig:conservation-hyp-inferred}
    \end{figure}
\end{frame}

\begin{frame}{Merger Resolution}
    \begin{figure}
        \centering
        \begin{subfigure}{0.3\textwidth}
            \centering
            \scalebox{0.65}{
                \begin{tikzpicture}[minimum size=58pt,scale=0.45, every node/.style={scale=0.45, font=\LARGE}, thick]
                    \input{images/conservation/division_detected.tex}
                \end{tikzpicture}
            }
        \end{subfigure}
        \hfill
        \begin{subfigure}{0.3\textwidth}
            \centering
            \scalebox{0.65}{
                \begin{tikzpicture}[minimum size=58pt,scale=0.45, every node/.style={scale=0.45, font=\LARGE}, thick]
                    \input{images/conservation/division_duplicated.tex}
                \end{tikzpicture}
            }
        \end{subfigure}
        \hfill
        \begin{subfigure}{0.3\textwidth}
            \centering
            \scalebox{0.65}{
                \begin{tikzpicture}[minimum size=58pt,scale=0.45, every node/.style={scale=0.45, font=\LARGE}, thick]
                    \input{images/conservation/division_inferred.tex}
                \end{tikzpicture}
            }
        \end{subfigure}
        \caption{Handling divisions.}
        \label{fig:conservation-hyp-inferred}
    \end{figure}
\end{frame}

\begin{frame}{Experimental Results}
    \begin{table}
        \centering
        \def\arraystretch{0.5}
\newlength\tablemathtext
\settototalheight\tablemathtext{\parbox{\linewidth}{$t=75$, $\id=446$}}
% \setlength{\tabcolsep}{1pt}
\scalebox{0.95}{
    \begin{tabular}{lccc}
        \toprule
        Time Step \& Id  & Raw Data & Segmentation & GMM fit \\ \midrule
        $t=75$, $\id=446$&
        \raisebox{\tablemathtext-\height}{
            \includegraphics[width=0.18\textwidth]{images/conservation/t=75,id=446,k=3_raw.png}} &
        \raisebox{\tablemathtext-\height}{
            \includegraphics[width=0.18\textwidth]{images/conservation/t=75,id=446,k=3_label.png}} &
        \raisebox{\tablemathtext-\height}{
            \includegraphics[width=0.18\textwidth]{images/conservation/t=75,id=446,k=3_fit.png}} \\
        &&& \\
        $t=75$, $\id=206$&
        \raisebox{\tablemathtext-\height}{
            \includegraphics[width=0.18\textwidth]{images/conservation/t=75,id=206,k=2_raw.png}} &
        \raisebox{\tablemathtext-\height}{
            \includegraphics[width=0.18\textwidth]{images/conservation/t=75,id=206,k=2_label.png}} &
        \raisebox{\tablemathtext-\height}{
            \includegraphics[width=0.18\textwidth]{images/conservation/t=75,id=206,k=2_fit.png}} \\
        &&& \\
        $t=85$, $\id=334$&
        \raisebox{\tablemathtext-\height}{
            \includegraphics[width=0.18\textwidth]{images/conservation/t=85,id=334,k=4_raw.png}} &
        \raisebox{\tablemathtext-\height}{
            \hspace{-3pt}\includegraphics[width=0.18\textwidth]{images/conservation/t=85,id=334,k=4_label.png}} &
        \raisebox{\tablemathtext-\height}{
            \includegraphics[width=0.18\textwidth]{images/conservation/t=85,id=334,k=4_fit.png}} \\
        \bottomrule
    \end{tabular}
}
\def\arraystretch{1.0}

% WHY THE FUCK THE NEED FOR NEGATIVE HSPACE IN LAST ROW?



%%% Local Variables: 
%%% mode: latex
%%% TeX-master: "../../main"
%%% End: 

        \caption{Inferred configuration of the subset.}
        \label{tab:conservation-gmm-fits}
    \end{table}
\end{frame}



%%% Local Variables: 
%%% mode: latex
%%% TeX-master: "../main"
%%% End: 
